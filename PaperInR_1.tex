\documentclass[11pt]{article}

\title{Replicate Paper 1 Meow!!!}
\author{
        Sony Suo\\
        Evans School of Public Policy and Governance\\
        University of Washington\\
        Seattle, WA 98105, \underline{United States}\\
        \texttt{suo1993@uw.edu}
}
\date{\today}


\usepackage{Sweave}
\begin{document}
\Sconcordance{concordance:PaperInR_1.tex:PaperInR_1.Rnw:%
1 13 1 1 0 39 1}


\maketitle


\begin{abstract}
This is an example on how to make a reproducible paper. We are using R from Rstudio, creating an RSweave document. This is a nice start to create a nice paper and get an A+. The next sections will show the steps taken.
\end{abstract}

\section{Introduction}\label{intro}
This is my intro to my great paper, I will explain the cool things I can do with my new `computational thinking' powers combined with some Latex.

This is my nice intro to my great paper, 
I will explain the cool things I can do with my new `computational thinking' powers combined with some Latex.



This is my nice intro to my great paper, 
I will explain the cool things 
I can do with my new `computational thinking' 
powers
combined with some Latex.

\section{Explaining Labels}\label{outline}

Sections may use a label\footnote{In fact, you can have a label wherever you think a future reference to that content might be needed.}. This label is needed for referencing. For example the next section has label \emph{datas}, so you can reference it by writing: As we see in section \ref{datas}.

\section{Data analysis}\label{datas}

Here you can explain how to get the data.

\subsection{Exploration}\label{eda}

Here, I start exploring the data. 

\subsection{Modeling}\label{model}

Here I can propose a regression model.
\end{document}
